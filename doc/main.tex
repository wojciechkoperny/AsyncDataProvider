\documentclass[table]{article}

\usepackage[english]{babel}
\usepackage{amsmath,amssymb}
\usepackage{parskip}
\usepackage{graphicx}
\usepackage{plantuml}

\usepackage{minted}
\newmintedfile[cppfile]{cpp}{frame=single,linenos,style=vs}

% Margins
\usepackage[top=2.5cm, left=3cm, right=3cm, bottom=4.0cm]{geometry}
% Colour table cells

% Get larger line spacing in table
\newcommand{\tablespace}{\\[1.25mm]}
\newcommand\Tstrut{\rule{0pt}{2.6ex}}         % = `top' strut
\newcommand\tstrut{\rule{0pt}{2.0ex}}         % = `top' strut
\newcommand\Bstrut{\rule[-0.9ex]{0pt}{0pt}}   % = `bottom' strut



%
%     Title
%
\title{Asynchronous Data Provider}
\author{Wojciech Koperny, Sławomir Cielepak}
\date{\today}

\begin{document}
\maketitle



%
%   System overview
%
\section{System overview}

The Asynchronous Data Provider system is a generic service that acts as a proxy to a simple key-value
database. Each entry in the database is indexed with unique Id. The backend that will handle such storage
is yet to be decided.

A provider exposes a C++ interface that can be used by external applications directly in the code. It is
provided as a shared or static library.

The main goal of the system is to provide concurrent access to the data for multiple threads, thus
maintaining the high responsiveness of the system.

Each data request is cached in the system and in case of cache miss, a request is passed on a TaskPool
queue. Tasks are consumed in parallel and provide results asynchronously directly to the client.

It is assumed that retrieval of data from the database might be occupied with high resource usage and be
time consuming.

\subsection{System objects}

The figure \ref{fig:system_overview} shows the main system objects.
\begin{figure}[h]
    \centering
    \resizebox{.6\linewidth}{!}{\begin{plantuml}
    @startuml

    interface "Data\nInterface" as DI
    actor Client

    package "Data Provider" {
        [Dispatcher] -down- DI

        [Data Cache] as DataCache
        [Task Pool] as TaskPool
        [Task]
        [Database]

        Dispatcher -left-> DataCache
        Dispatcher -right-> TaskPool
        Dispatcher -up-> Task
        TaskPool -up-> Task
        Task -right-> Database
        Task -> DataCache
    }

    Client -right-(DI

    @enduml
\end{plantuml}
}
    \caption{Asynchronous Data Provider system overview}
    \label{fig:system_overview}
\end{figure}


\subsubsection{Dispatcher}

A Dispatcher is an entity that provides boundary interface for the system receiving the Data Requests form
the Client. For every request a \mintinline{cpp}{std::future<T>} object is returned.

The Dispatcher first tries to get the requested data from Cache, if it's available, it sets the
\mintinline{cpp}{std::promise<T>} immediately. In case the data is not available, Dispatcher creates a new
Task and puts in on the TaskPool queue.

\subsubsection{Data Cache}

<description>

\subsubsection{Task Pool}

<description>

\subsubsection{Database}

<description>



%
% Use Cases
%
\pagebreak
\section{Use cases}

\subsubsection{Client requests data by ID}

<description>

\begin{figure}[ht]
    \centering
    \begin{plantuml}
    @startuml

    actor Client
    participant "Data Interface" as di
    participant "Data Cache" as data_cache
    participant "Task Pool" as task_pool

    Client -> di : RequestDataById(id: uint)
    activate di

    di -> di :: create promise

    di -> data_cache : getData()
    activate data_cache

    data_cache -> data_cache : create optional<Data>

    alt "cache hit"
    data_cache -> data_cache : optional::emplace(data)
    end

    di <-- data_cache : optional<Data>
    deactivate data_cache

    alt "data available"
    di -> di : promise::set_value(Data)
    else
    di -> Task ** : new Task(Promise)

    di ->> task_pool : enqueue_task(Task)
    activate task_pool
    end

    Client <-- di : return promise::get_future()
    deactivate di
    hnote over di : Idle

    task_pool -> Task : startTask()
    deactivate task_pool
    activate Task

    Task -> Database : getData()
    activate Database
    deactivate Database

    Task -> Task : ComputeData()
    Task -> Task : promise::set_value(Data)
    activate Task
    Task ->> Client : future::ready
    deactivate Task

    Task -> data_cache : putData(Data)
    activate data_cache
    deactivate data_cache
    deactivate Task
    destroy Task


    Client -> Client : future::get_value()
    activate Client
    deactivate Client

    @enduml
\end{plantuml}

    \caption{PlantUML diagram from file}
    \label{fig:client_requests_data}
\end{figure}




%
% Interfaces
%
\pagebreak
\section{Interfaces}

<description>

\begin{listing}[ht]
    \cppfile{Listings/hello_world.cpp}
    \caption{Example source code from external file}
    \label{lst:hello_world}
\end{listing}

\end{document}
